\documentclass{beamer}
\usetheme[hideothersubsections]{HRTheme}
\usepackage{beamerthemeHRTheme}
\usepackage{graphicx}
\usepackage[space]{grffile}
\usepackage{listings}
\usepackage{animate}
\lstset{language=SQL,
basicstyle=\ttfamily\footnotesize,
mathescape=true,
keywordstyle=\color{blue},
breaklines=true,
showspaces=false,
showstringspaces=false}
\usepackage[utf8]{inputenc}
\usepackage{color}
\newcommand{\red}[1]{
\textcolor{red}{#1}
}
\newcommand{\ts}{\textbackslash}

\title{ACID, Transaction Management and Conccurrency Control }

\author{ }

\institute{Hogeschool Rotterdam \\ 
Rotterdam, Netherlands}

\date{}

\begin{document}
\maketitle

\SlideSection{Introduction}
\SlideSubSection{Lecture topics}
\begin{slide}{
\item ACID.
\item Transaction Management.
\item Concurrency Control.
}\end{slide}

\SlideSection{Transactions and Concurrency}
\SlideSubSection{Reasons}
\begin{slide}{
\item \emph{Concurrent} execution of user programs is essential for good DBMS performance.
Because disk accesses are frequent, and relatively slow, it is important to keep the cpu busy by working on several user programs concurrently.
\item A user’s program may carry out many operations on the data retrieved from the database, but the DBMS is only concerned about what data is read/written from/to the database.
\item A \emph{transaction} is the DBMS’s abstract view of a user program:  a sequence of \red{reads} and \red{writes}.
}\end{slide}


\SlideSubSection{Transactions and Concurrency}
\begin{slide}{
\item Users submit transactions, and can think of each transaction as executing by itself.
\begin{itemize}
	\item \emph{Concurrency} is achieved by the DBMS, which interleaves actions (reads/writes of DB objects) of various transactions.
	\item transaction must leave the database in a consistent state if the DB is consistent when the transaction begins.
	\item Beyond this, the DBMS does not really understand the semantics of the data.  (e.g., it does not understand how the interest on a bank account is computed).	
\end{itemize}
\item Issues:  Effect of interleaving transactions, and crashes.
}\end{slide}

\SlideSubSection{The ACID properties }
\begin{slide}{
\item A RDBS ensures this four properties of a transaction:
\begin{itemize}
	\item Atomicity: states that database modifications must follow an all or nothing rule.
	\pause
	\item Consistency: states that only valid data will be written to the database.
	\pause
	\item Isolation: requires that multiple transactions occurring at the same time not impact each others execution.
	\pause
	\item Durability: ensures that any transaction committed to the database will not be lost. 	
\end{itemize}
}\end{slide}

\SlideSubSection{Atomicity of Transactions}
\begin{slide}{
\item A transaction might \red{commit} after completing all its actions, or it could \red{abort} (or be aborted by the DBMS) after executing some actions.
\item A very important property guaranteed by the DBMS for all transactions is that they are atomic.  That is, a user can think of a transaction as always executing all its actions in one step, or not executing any actions at all.
\item DBMS logs all actions so that it can undo the actions of aborted transactions.		
}\end{slide}

\SlideSection{Example of a Transaction}
\begin{frame}[fragile]{Transaction}
\begin{lstlisting}
BEGIN;
UPDATE ships SET name = 'Alpha'
	WHERE name = 'Oleg';
SAVEPOINT my_savepoint;
UPDATE ships SET integrity = 30
	WHERE name = 'Alpha';
-- oops ... forget to update also Beta 
ROLLBACK TO my_savepoint;
UPDATE ships SET integrity = integrity + 10
	WHERE name = 'Beta';
COMMIT;		
\end{lstlisting}
\red{When are those updates valid states for other actions?}  
\end{frame}	


\SlideSubSection{Transactions}
\begin{slide}{
\item We simplify transaction queries for readability in our next example  
\pause
\item Consider those two transactions:
\begin{itemize}
	\item T1:	BEGIN   A=A+100,   B=B-100   END
	\item T2:	BEGIN   A=1.06*A,   B=1.06*B   END
	
\end{itemize}
\item The first transaction is transferring \$100 from B’s account to A’s account.  The second is crediting both accounts with a 6\% interest payment.
\item \red{How are those transaction scheduled?}	
}\end{slide}


\SlideSubSection{Transactions}
\begin{slide}{
\item possibility one 
\begin{itemize}
	\item T1:A=A+100,\hspace{2cm}B=B-100   
	\item T2:\hspace{2cm} A=1.06*A,\hspace{2cm}B=1.06*B
\end{itemize}
\item possibility two
\begin{itemize}
	\item T1:A=A+100,\hspace{4cm}B=B-100   
	\item T2:\hspace{2cm}A=1.06*A, B=1.06*B
\end{itemize}
\item The DBMS’s view of the second schedule:
\begin{itemize}
	\item T1:R(A),W(A),\hspace{4cm}   		     	       R(B), W(B)
	\item T2:\hspace{2cm}R(A), W(A), R(B), W(B)
\end{itemize}
\red{R(A): read A, W(A): write A}
}\end{slide}


\end{document}

\begin{slide}{
\item ...
}\end{slide}

\begin{frame}[fragile]
\begin{lstlisting}
...
\end{lstlisting}
\end{frame}

\begin{tabular}{|c|c|c|}
	\hline
	\multicolumn{3}{|c|}{\textbf{Projectiles}} \\
	\hline
	target & position & damage \\
	\hline
	Red 3 & (0,1,0) & 30 \\
	\hline
	Red 3 & (3,1,-2) & 50 \\
	\hline
	Red 3 & (0,2.5,1) & 100 \\
	\hline
\end{tabular} 
\begin{lstlisting}
SELECT s.name,p.damage
FROM ships s,projectiles p
WHERE s.position = p.position AND
s.name = p.target
\end{lstlisting}
