\documentclass{beamer}
\usetheme[hideothersubsections]{HRTheme}
\usepackage{beamerthemeHRTheme}
\usepackage{graphicx}
\usepackage[space]{grffile}
\usepackage{listings}
\lstset{language=C,
basicstyle=\ttfamily\footnotesize,
mathescape=true,
breaklines=true}
\usepackage[utf8]{inputenc}
\usepackage{color}
\newcommand{\red}[1]{
\textcolor{red}{#1}
}
\newcommand{\ts}{\textbackslash}

\title{E-R model, Relational model, SQL}

\author{ }

\institute{Hogeschool Rotterdam \\ 
Rotterdam, Netherlands}

\date{}

\begin{document}
\maketitle

\SlideSection{Introduction}
\SlideSubSection{Lecture topics}
\begin{slide}{
\item E-R model.
\item Relational model.
\item SQL, and examples.
}\end{slide}

\SlideSection{E-R model}
\SlideSubSection{Overview}
\begin{slide}{
\item Highest level of database modelling.
\item Model the conceptual aspect of the database.
\item Far from the physical representation in the DBMS.
}\end{slide}

\SlideSubSection{Entity}
\begin{slide}{
\item Anything which can exist on its own on the database
\item Consider a database for a space shooter game
\item Starships, asteroids are entities, they have a meaning on their own

%insert picture here
}\end{slide}

\SlideSubSection{Attributes}
\begin{slide}{
\item They model characteristics of the entity.
\item \textbf{Starship:} velocity, shield, armour, weapon, [...]
\item \textbf{Asteroid:} velocity, mass, integrity, [...]

%insert picture here
}\end{slide}

\SlideSubSection{Relations}
\begin{slide}{
\item They describe the associations among entities (two or more).
\item They have a cardinality: number of participants for each side.

%insert picture here
}\end{slide}

\SlideSubSection{Relations - 1 : 1}
\begin{slide}{
\item Entity modelling a pilot and one modelling a starship.
\item Related by ``drives''.
\item The cardinality is 1:1 : one pilot drives at most one starship, and one starship can contain only one pilot.

%insert picture here
}\end{slide}

\SlideSubSection{Relations - 1 : N}
\begin{slide}{
\item Entity modelling a starship and one modelling a weapon.
\item Realted by ``mounted''
\item The cardinality is 1:N : a weapon can be mounted only on one starship, but a starship can mount more than one weapon.

%insert picture here
}\end{slide}

\SlideSubSection{Relations - N : M}
\begin{slide}{
\item Entity modelling a starship and one modelling an asteroid.
\item Realted by ``collides with''
\item The cardinality is N : M : several starships can collide with several asteroids.

%insert picture here
}\end{slide}

\SlideSubSection{Keys}
\begin{slide}{
\item A way to uniquely identify an entity.
\item A key is a set of attributes that have unique values among entities.
\item \textbf{Starship:} Serial number.

%insert picture here
}\end{slide}

\SlideSubSection{Weak entities}
\begin{slide}{
\item Entities which do not have a key attribute.
\item \textbf{Asteroids:} There can be two asteroids with the same position, same mass, velocity, etc.

%insert picture here
}\end{slide}

\SlideSection{Relational model}
\SlideSubSection{Overview}
\begin{slide}{
\item ---
}\end{slide}


\end{document}

\begin{slide}{
\item ...
}\end{slide}

\begin{frame}[fragile]
\begin{lstlisting}
...
\end{lstlisting}
\end{frame}
