\section*{Exam sample}
What follows is a concrete example of the exam.


\paragraph{Question I: formal rules} \ \\

\textit{You start at point (0,0). Take a step in the direction (10,0) until you are above point (45,0). Then take five steps in the direction (0,2). Where do you end up?}

\ \\ 

\textbf{Answer:} \textit{The trajectory is:}

\begin{lstlisting}

P1 = (50,0)
 +----- P2 = (50,10)
 |
 |
 |
 |
P0 = (0,0)
\end{lstlisting}

\ \\ 

\textbf{Points:} \textit{25\%.}

\ \\ 
\ \\ 

\paragraph{Question II: program state} \ \\ 

\textit{Fill-in the program state with the values that the variables assume while running the sample below.}

\begin{lstlisting}
y = 1
for i in range(0, 5):
    y = y * 2
\end{lstlisting}

\ \\ 

\textbf{Answer:} \textit{The variable allocations are:}

\begin{tabular}{| c | c | c | c | c | c | c | c |}
\hline
\textbf{y} & 1 & 1 & 2 & 4 & 8 & 16 & 32 \\
\hline
\textbf{i} & n.a. & 0 & 1 & 2 & 3 & 4 & 4 \\
\hline
\end{tabular}

\ \\
s
\textbf{Points:} \textit{25\%.}

\ \\ 

\textbf{Grading:} \textit{Full points if all values are correctly listed in the right order. Half points if at least half of values are listed in the right order. Zero points otherwise.}

\ \\ 

\textbf{Associated learning goals:} \texttt{CMC}.

\ \\ 

\paragraph{Question III: variables, expressions, and data types}

\ \\ 

\textit{What is the value and the type of all variables after execution of the following code?}
\begin{lstlisting}
v = 0
i = "Hello + world"
j = "Hello" + "world"
k = 10 / 3
\end{lstlisting}

\ \\ 

\textbf{Answer:} \textit{The value and type of all variables after execution is:}

\begin{tabular}{| l | c | c | }
\hline
\textbf{Variable} & \textbf{Value} & \textbf{Type} \\
\hline
v & 0 & int \\
\hline
i & 'Hello + world' & str \\
\hline
j & 'Helloworld' & str \\
\hline
k & 3 & int \\
\hline
l & 3.3333$\dots$ & float \\
\hline
\end{tabular}

\ \\ 

\textbf{Points:} \textit{25\%.}

\ \\ 

\textbf{Grading:} \textit{All values and types are correct: full-points. At least half the values and at least half the types are correct: half points. Zero points otherwise.}

\ \\ 

\textbf{Associated learning goals:} \texttt{VAR}, \texttt{EXPR}.

\ \\ 

\paragraph{Question IV: control flow}

\ \\ 

\textbf{General shape of the question:} \textit{What is the value of all variables after execution of the following code?}

\ \\ 

\textbf{Concrete example of question:} \textit{What is the value of all variables after execution of the following code?}

\begin{lstlisting}
v = 0
for i in range(1,15):
  if (i % 2 == 0) & (i % 3 == 0):
    v = v + i
\end{lstlisting}

\ \\ 

\textbf{Concrete example of answer:} \textit{The value of all variables after execution is:}

\begin{tabular}{| l | c |}
\hline
\textbf{Variable} & \textbf{Value} \\
\hline
\texttt{i} & \texttt{14} \\
\hline
\texttt{v} & \texttt{18} \\
\hline
\end{tabular}

\ \\ 

\textbf{Points:} \textit{25\%.}

\ \\ 

\textbf{Grading:} \textit{All values are correct: full-points. At least half the values are correct: half points. Zero points otherwise.}

\ \\ 

\textbf{Associated learning goals:} \texttt{COND}, \texttt{LOOP}.

\ \\ 
