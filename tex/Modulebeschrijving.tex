\section*{Modulebeschrijving}
\begin{tabularx}{\textwidth}{|>{\columncolor{lichtGrijs}} p{.26\textwidth}|X|}
	\hline
	\textbf{Module name:} & \modulenaam\\
	\hline
	\textbf{Module code: }& \modulecode\\
	\hline
	\textbf{Study points \newline and hours of effort for full-time students:} & This module gives \stdPunten, in correspondance with 112 hours:
	\begin{itemize}
		\item 3 x 7 hours frontal lecture (2 x 7 for part-time students)
		\item 9 x 7 hours self-study for the practicum (10 x 7 for part-time students)
		\item the rest is self-study for the theory
	\end{itemize} \\
	\hline
	\textbf{Examination:} & Written examination and practicums (with oral check) \\
	\hline
	\textbf{Course structure:} & Lectures, self-study, and practicums \\
	\hline
	\textbf{Prerequisite knowledge:} & None. \\
	\hline
	\textbf{Learning tools:}  &
		\begin{itemize}
			\item Book: Database management systems (3rd edition); authors Ramakrishnan and Gehrke
			\item Book: NO SQL Distilled; authors Sadalage and Fowler
			\item Presentations (in pdf): found on N@tschool and on the GitHub repository \url{https://github.com/hogeschool/INFDEV03-5}
			\item Assignments, to be done at home (pdf): found on N@tschool and on the GitHub repository \url{https://github.com/hogeschool/INFDEV03-5}
		\end{itemize} \\
	\hline
	\textbf{Connected to \newline competences:} &
		\begin{itemize}
			\item Analysis, design, and realisation of software at level 2
		\end{itemize} \\
	\hline
	\textbf{Learning objectives:} &
		At the end of the course, the student can:
			\begin{itemize}
				\item \textbf{realise} a database in a modern, standard, SQL-based RDBMS \texttt{RDBMS}
				\item \textbf{realise} integration between an OO application and a DBMS through an ORM mapping \texttt{ORM}

				\item \textbf{analyse} and optimize performance of a queries \texttt{OPT}
				\item \textbf{describe} the differences between relational and non-relational databases \texttt{NON-REL}

				\item \textbf{describe} models of concurrency and transactions in a modern DBMS \texttt{TRANS-CONS}

			\end{itemize} \\
		
	\hline
\end{tabularx}
\newpage

\begin{tabularx}{\textwidth}{|>{\columncolor{lichtGrijs}} p{.26\textwidth}|X|}
	\hline
	\textbf{Content:}&
	\begin{itemize}
		\item relevant concepts in relational databases and ORM's
		\item performance issues with some query operators (\texttt{where}, \texttt{join}, etc.)
		\item indexes and their role in performance
		\item fundamental properties of DBMS's: atomicity, consistency, isolation, and durability (ACID)
		\item concurrency and transactions
		\item crash recovery
		\item other tradeoffs than ACID and the CAP theorem: basic availability, soft state, and eventual consistency (BASE)
		\item some BASE DBMS's: noSQL
	\end{itemize} \\
	\hline
	\textbf{Course owners:} & \author\\
	\hline
	\textbf{Date:} & \today \\
	\hline
\end{tabularx}
\newpage
