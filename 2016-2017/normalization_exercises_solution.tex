\documentclass[10pt,a4paper]{article}
\usepackage[utf8]{inputenc}
\usepackage{amsmath}
\usepackage{amsfonts}
\usepackage{amssymb}
\usepackage{float}
\usepackage{graphicx}
\usepackage{pdfpages}
\usepackage{multirow}
\usepackage{enumitem}
\author{Francesco Di Giacomo, Ahmad Omar}
\date { }
\title{Normalization exercises}
\usepackage{listings}
\lstset{language=SQL,
basicstyle=\ttfamily\footnotesize,
frame=shadowbox,
mathescape=true,
showstringspaces=false,
showspaces=false,
showstringspaces = false,
breaklines=true}
\newcommand{\tu}{\textunderscore}
\newcommand{\valseq}[1]{$\lbrace$ #1 $\rbrace$}
\newcommand{\tuple}[2]{$t_{#1}$[#2]}
\newcommand{\fdep}[2]{#1 $\rightarrow$ #2}
\newcommand{\normalization}[3]{
	
	\vspace{0.3cm}
	\noindent
	1NF:\\
	#1
	
	\vspace{0.3cm}
	\noindent
	2NF:\\
	#2
	
	\vspace{0.3cm}
	\noindent
	BCNF:\\
	#3
	}

\begin{document}
	\maketitle
	
	\textbf{Note:} For each of the following exercises complete the following tasks:
	\begin{enumerate}
		\item (Complete after lesson 1) Find what normal form each of the following tables satisfies.Motivate the answer according the definition of normal forms seen in class.
		\item (Complete after lesson 2) Apply the normalization algorithms seen in class to each table. Use intermediate refinements, i.e. if the table is in 1NF first normalize in 2NF and, if necessary, in BCNF.
	\end{enumerate}
	
	
	\paragraph*{Exercise 1 - Lockers}
	
	The table is in no normal form, since it contains a multi-value attribute.
	
	\begin{table}[!h]
		\centering
		\begin{tabular}{|c|c|c|}
			\hline
			\multicolumn{3}{|c|}{\textbf{teacher}} \\
			\hline
			\underline{teacher\tu id} & name & surname \\
			\hline
		\end{tabular}
		
		\vspace{0.5cm}
		\begin{tabular}{|c|c|c|c|}
			\hline
			\multicolumn{4}{|c|}{\textbf{locker}} \\
			\hline
			\underline{id} & \underline{teacher\tu id} & key\tu num & size \\
			\hline
		\end{tabular}
	\end{table}
	
	\noindent
	In \texttt{locker} the attribute \texttt{teacher\textunderscore id} is a foreign key to \texttt{teacher}.
	The table is already in BCNF since all dependencies have the left argument which is a superkey.
	
	\paragraph*{Exercise 2 - Library}
	
	The table is in no normal form since it contains a multi-value attribute.
	
	\normalization{
	\begin{table}[!h]
		\begin{tabular}{|c|c|c|}
			\hline
			\multicolumn{3}{|c|}{\textbf{library}} \\
			\hline
			\underline{card\tu num} & name & surname \\
			\hline
		\end{tabular}
	\end{table}
		\begin{table}[!h]
			\centering
			\begin{tabular}{|c|c|c|c|c|c|}
				\hline
				\multicolumn{6}{|c|}{\textbf{borrowed\tu books}}\\
				\hline
				\underline{author} & \underline{car\tu num} & \underline{title} & \underline{date} & return\tu date \\
				\hline
			\end{tabular}
		\end{table}
		
		\noindent
		In \texttt{member} the attribute card\tu num is a foreign key to \texttt{library}.
	}
		
	{
		The tables rreated at the previous step are in 2NF because no functional dependencies exist.
	}
	{
		All the tables at this point are in BCNF because all the dependencies have a superkey as left argument.}
	
	
	
	\paragraph*{Exercise 3 - Books}
	
	\normalization{
		The table is already in 1NF because all the attributes are atomic.}
	{
		Both dependencies have the left argument that is part of a key and the right argument that is a non-key attribute, thus they break 2NF
		
		\begin{table}[!h]
			\centering
			\begin{tabular}{|c|c|}
				\hline
				\multicolumn{2}{|c|}{\textbf{book}}\\
				\hline
				\underline{author} & \underline{title}\\
				\hline
			\end{tabular}
			
			\vspace{0.5cm}
			\begin{tabular}{|c|c|}
				\hline
				\multicolumn{2}{|c|}{\textbf{book\textunderscore author}}\\
				\hline
				\underline{author} & author\textunderscore bdate\\
				\hline
			\end{tabular}
			
			\vspace{0.5cm}
			\begin{tabular}{|c|c|c|c|}
				\hline
				\multicolumn{4}{|c|}{\textbf{description}}\\
				\hline
				\underline{title} & genre & page & section\\
				\hline
			\end{tabular}
		\end{table}
		\noindent
		In the table \texttt{book} the attribute \textunderscore{author} is a foreign key to \texttt{book\textunderscore author} and \texttt{title} is a foreign key to \texttt{description}.}{
		All the tables are already in BCNF since the left argument of all the dependencies are superkeys.}
	
	\paragraph*{Exercise 4 - Houses}
	
	The table is in no normal form because it contains a multi-value attribute.
	
	\normalization{
		\begin{table}[!h]
			\centering
			\begin{tabular}{|c|c|c|c|c|}
				\hline
				\multicolumn{5}{|c|}{\textbf{houses}} \\
				\hline
				\underline{owner} & \underline{postal\tu code}  & price & size & account\\ [0.3cm]
				\hline
			\end{tabular}
		\end{table}
		\begin{table}[!h]
			\centering
			\begin{tabular}{|c|c|c|c|c|}
				\hline
				\multicolumn{5}{|c|}{\textbf{owner\tu address}} \\
				\hline
				\underline{owner} & \underline{postal\tu code}  & \underline{city} & \underline{street} & \underline{number}\\ [0.3cm]
				\hline
			\end{tabular}
		\end{table}
		
		\noindent
		In \texttt{owner\tu address} the attribute \texttt{postal\tu code} is a foreign key to \texttt{houses}
		}{
			Using the decomposition rule on \fdep{\texttt{postal\textunderscore code}}{\texttt{price, size}} we get the functional dependencies \fdep{\texttt{postal\textunderscore code}}{\texttt{price}} and \fdep{\texttt{postal\textunderscore code}}{\texttt{size}}. Both break the 2NF because they are partially dependent on the primary key. Also the dependency \fdep{\texttt{owner}}{\texttt{account}} breaks 2NF.
			
			\begin{table}[!h]
				\centering
				\begin{tabular}{|c|c|}
					\hline
					\multicolumn{2}{|c|}{\textbf{account}} \\
					\hline
					\underline{owner} & account\\[0.3cm]
					\hline
				\end{tabular}
				
				\vspace{0.5cm}
				\begin{tabular}{|c|c|c|}
					\hline
					\multicolumn{3}{|c|}{\textbf{property}} \\
					\hline
					\underline{postal\textunderscore code} & price & size\\[0.3cm]
					\hline
				\end{tabular}
				
				\vspace{0.5cm}
				\begin{tabular}{|c|c|c|c|c|}
					\hline
					\multicolumn{5}{|c|}{\textbf{owner\tu address}} \\
					\hline
					\underline{owner} & \underline{postal\tu code}  & \underline{city} & \underline{street} & \underline{number}\\ [0.3cm]
					\hline
				\end{tabular}
			\end{table}
			
			\noindent
			In \texttt{houses} the attribute \texttt{owner} is a foreign key to \texttt{account} and \texttt{postal\tu  code} is a foreign key to \texttt{property}
		}
		{
			All tables are already in BCNF because the left argument of all the dependencies is a superkey.
		}
	
	\paragraph*{Exercise 5 - Port}
	\normalization{
		The table is already in 1NF because all attributes are atomic}{
		The dependencies \fdep{\texttt{ship\textunderscore name}}{\texttt{docked\textunderscore at,country,weight,class}} and \fdep{\texttt{captain}}{\texttt{cpt\textunderscore license}} break the 2NF because the left argument is part of a key and the right argument is a non-key attribute (again use the decomposition rule). Note that the dependency \fdep{\texttt{docked\textunderscore at}}{\texttt{country}} does not break 2NF because the left argument is not part of any key (the 2NF only considers dependencies where the left side is part of a key)
		
		\begin{table}[!h]
			\centering
			\begin{tabular}{|c|c|}
				\hline
				\multicolumn{2}{|c|}{\textbf{ship\textunderscore captain}}\\
				\hline
				\underline{ship\textunderscore name} & \underline{captain}\\[0.3cm]
				\hline
			\end{tabular}
			
			\vspace{0.5cm}
			\begin{tabular}{|c|c|c|c|c|}
				\hline
				\multicolumn{5}{|c|}{\textbf{ship}}\\
				\hline
				\underline{ship\textunderscore name} & weight & class & docked\textunderscore at & country\\[0.3cm]
				\hline
			\end{tabular}
			
			\vspace{0.5cm}
			\begin{tabular}{|c|c|}
				\hline
				\multicolumn{2}{|c|}{\textbf{license}}\\
				\hline
				\underline{captain} & cpt\textunderscore license\\ [0.3cm]
				\hline
			\end{tabular}
			
			In \texttt{ship\textunderscore captain} the attribute \texttt{ship\textunderscore name} is foreign key to \texttt{ship} and \texttt{captain} is a foreign key to \texttt{license}.
		\end{table}}{
			The dependency \fdep{\texttt{docked\textunderscore at}}{\texttt{country}} breaks BCNF because the left argument is not a superkey. The table \texttt{ship} must be decomposed in BCNF.
			
			\begin{table}[!h]
				\centering
				\begin{tabular}{|c|c|c|c|}
					\hline
					\multicolumn{4}{|c|}{\textbf{ship}}\\
					\hline
					\underline{ship\textunderscore name} & weight & class & docked\textunderscore at\\ [0.3cm]
					\hline
				\end{tabular}
				
				\vspace{0.5cm}
				\begin{tabular}{|c|c|}
					\hline
					\multicolumn{2}{|c|}{\textbf{docking}}\\
					\hline
					\underline{docked\textunderscore at} & country\\
					\hline
				\end{tabular}
			\end{table}
			
			\noindent
			In the table \texttt{ship} the attribute \texttt{docked\textunderscore at} is a foreign key to \texttt{docking}.
		}
	
	\paragraph*{Exercise 6 - Cellar}
	\normalization{
		The table is in 1NF because it contains only atomic attributes}
	{
		The dependencies \fdep{\texttt{producer}}{\texttt{country,location}} and \fdep{\texttt{wine}}{\texttt{bottling\textunderscore date,price/l,grape\textunderscore variety}} break the 2NF since the left side is part of a key and the right side is a non-key attribute. The other dependencies do not break 2NF because they do not have a left argument that is part of a key
		
		\newpage
		
		\begin{table}[!h]
			\centering
			\begin{tabular}{|c|c|}
				\hline
				\multicolumn{2}{|c|}{\textbf{cellar}}\\
				\hline
				\underline{producer} & \underline{wine}\\
				\hline
			\end{tabular}
			
			\vspace{0.5cm}
			\begin{tabular}{|c|c|c|}
				\hline
				\multicolumn{3}{|c|}{\textbf{producer}}\\
				\hline
				\underline{producer} & country & location\\
				\hline
			\end{tabular}
			
			\vspace{0.5cm}
			\begin{tabular}{|c|c|c|c|}
				\hline
				\multicolumn{4}{|c|}{\textbf{wine}}\\
				\hline
				\underline{wine} & bottling\textunderscore date & price/l & grape\textunderscore variety\\
				\hline
			\end{tabular}
		\end{table}
		
		\noindent
		In \texttt{cellar} the attribute \texttt{producer} is a foreign key to the table \texttt{producer} and the attribute \texttt{wine} is a foreign key to the table \texttt{wine}.
	}
	{
		The dependencies \fdep{\texttt{location}}{\texttt{country}}
		\fdep{\texttt{grape\textunderscore variety}}{\texttt{price/l}} break BCNF because their left argument is not a superkey. The tables \texttt{wine} and \texttt{producer} must be normalized in BCNF.
		
		\begin{table}[!h]
			\centering
			
			\vspace{0.5cm}
			\begin{tabular}{|c|c|}
				\hline
				\multicolumn{2}{|c|}{\textbf{producer}}\\
				\hline
				\underline{producer} & location\\[0.3cm]
				\hline
			\end{tabular}
			
			\vspace{0.5cm}
			\begin{tabular}{|c|c|}
				\hline
				\multicolumn{2}{|c|}{\textbf{location}}\\
				\hline
				\underline{location} & country\\
				\hline
			\end{tabular}
			
			\vspace{0.5cm}
			\begin{tabular}{|c|c|c|}
				\hline
				\multicolumn{3}{|c|}{\textbf{wine}}\\
				\hline
				\underline{wine} & bottling\textunderscore date & grape\textunderscore variety\\
				\hline
			\end{tabular}
			
			\vspace{0.5cm}
			\begin{tabular}{|c|c|}
				\hline
				\multicolumn{2}{|c|}{\textbf{grape}}\\
				\hline
				\underline{grape\textunderscore variety} & price/l\\[0.3cm]
				\hline
			\end{tabular}		
		\end{table}
		
		\noindent
		In the table \texttt{wine} the attribute \texttt{grape\textunderscore variety} is foreign key to \texttt{grape} and in the table \texttt{producer} the attribute \texttt{location} is foreign key to the table \texttt{location}.
	}
	
	\newpage
	\Large
	\noindent
	\textbf{Note:} The following exercises are analogous to the previous exercises so only the final solution can be found
	
	\normalsize
	\paragraph*{Exercise 7 - Courses}
		
	In \texttt{teaching} the attribute \texttt{employee\textunderscore code} is a foreign key to \texttt{employee} and the attribute \texttt{course\textunderscore code} is a foreign key to \texttt{course}. In the table \texttt{course} the attribute study\textunderscore points is a foreign key to \texttt{hours}.

	\begin{table}[!h]
		\centering
		\begin{tabular}{|c|c|}
			\hline
			\multicolumn{2}{|c|}{\textbf{teaching}}\\
			\hline
			\underline{employee\textunderscore code} & \underline{course\textunderscore code}\\[0.3cm]
			\hline
		\end{tabular}
		
		\vspace{0.5cm}
		\begin{tabular}{|c|c|c|}
			\hline
			\multicolumn{3}{|c|}{\textbf{employee}}\\
			\hline
			\underline{employee\textunderscore code} & name & surname\\ [0.3cm]
			\hline
		\end{tabular}
		
		\vspace{0.5cm}
		\begin{tabular}{|c|c|c|}
			\hline
			\multicolumn{3}{|c|}{\textbf{course}}\\
			\hline
			\underline{course\textunderscore code} & course\textunderscore name & study\textunderscore points\\ [0.3cm]
			\hline
		\end{tabular}
		
		\vspace{0.5cm}
		\begin{tabular}{|c|c|}
			\hline
			\multicolumn{2}{|c|}{\textbf{hours}}\\
			\hline
			\underline{study\textunderscore points} & hours\\
			\hline
		\end{tabular}
	\end{table}
	
	
	\paragraph*{Exercise 8 - Flights}
	
	
	In the table \texttt{flying} the attribute \texttt{flight\textunderscore code} is a foreign key to \texttt{flight}, and the attribute \texttt{captain\textunderscore code} is a foreign key to \texttt{captain}. In the table \texttt{flight} the attributes \valseq{\texttt{departure,arrival}} are a foreign key to \texttt{duration}.
	\begin{table}[!h]
		\centering
		
		\begin{tabular}{|c|c|}
			\hline
			\multicolumn{2}{|c|}{\textbf{fying}}\\
			\hline
			\underline{flight\textunderscore code} & \underline{captain\textunderscore code}\\[0.3cm]
			\hline
		\end{tabular}
		
		\vspace{0.5cm}
		\begin{tabular}{|c|c|c|c|}
			\hline
			\multicolumn{4}{|c|}{\textbf{flight}}\\
			\hline
			\underline{flight\textunderscore code} & plane\textunderscore model & departure & arrival\\[0.3cm]
			\hline
		\end{tabular}
		
		\vspace{0.5cm}
		\begin{tabular}{|c|c|}
			\hline
			\multicolumn{2}{|c|}{\textbf{captain}}\\
			\hline
			\underline{captain\textunderscore code} & captain\textunderscore name\\[0.3cm]
			\hline
		\end{tabular}
		
		\vspace{0.5cm}
		\begin{tabular}{|c|c|c|}
			\hline
			\multicolumn{3}{|c|}{\textbf{duration}}\\
			\hline
			\underline{departure} & \underline{arrival} & flight\textunderscore time\\
			\hline
		\end{tabular}
	\end{table}
	
	\newpage
	\paragraph*{Exercise 9 - Buildings}
	
	In \texttt{materials} the attribute \texttt{building\textunderscore type\textunderscore code} is a foreign key to \texttt{building} and \texttt{material} is a foreign key to the table \texttt{material}. In the table \texttt{building} the attributes \valseq{\texttt{length,width,height}} are a foreign key to \texttt{taxes}.
	
	\begin{table}[!h]
		\centering
		
		\begin{tabular}{|c|c|c|}
			\hline
			\multicolumn{3}{|c|}{\textbf{building\textunderscore materials}}\\
			\hline
			\underline{building\textunderscore code} & \underline{component\textunderscore type} & \underline{material}\\[0.3cm]
			\hline
		\end{tabular}
		
		\vspace{0.5cm}
		\begin{tabular}{|c|c|c|}
			\hline
			\multicolumn{3}{|c|}{\textbf{material}}\\
			\hline
			\underline{material} & aximum\textunderscore pressure & specific\textunderscore weight\\[0.3cm]
			\hline
		\end{tabular}
		
		\vspace{0.5cm}
		\begin{tabular}{|c|c|c|c|}
			\hline
			\multicolumn{4}{|c|}{\textbf{building}}\\
			\hline
			\underline{building\textunderscore code} &  building\textunderscore length & building\textunderscore width & building\textunderscore height\\
			\hline
		\end{tabular}
		
		\vspace{0.5cm}
		\begin{tabular}{|c|c|c|c|}
			\hline
			\multicolumn{4}{|c|}{\textbf{taxes}}\\
			\hline
			\underline{building\textunderscore length} & \underline{building\textunderscore width} & \underline{building\textunderscore height} & tax\textunderscore rate\\[0.3cm]
			\hline
		\end{tabular}
	\end{table}

\end{document}
