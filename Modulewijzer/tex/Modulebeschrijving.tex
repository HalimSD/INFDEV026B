\section*{Modulebeschrijving}
\begin{tabularx}{\textwidth}{|>{\columncolor{lichtGrijs}} p{.26\textwidth}|X|}
	\hline
	\textbf{Module name:} & \modulenaam\\
	\hline
	\textbf{Module code: }& \modulecode\\
	\hline
	\textbf{Study points \newline and hours of effort for full-time students:} & This module gives \stdPunten, in correspondance with 84 hours:
	\begin{itemize}
		\item 2 x 8 hours frontal lecture.
		\item the rest is self-study for the theory and practicum.
	\end{itemize} \\
	\hline
	\textbf{Examination:} & Written examination and practicum (with oral check) \\
	\hline
	\textbf{Course structure:} & Lectures, self-study, and practicum \\
	\hline
	\textbf{Prerequisite knowledge:} & None. \\
	\hline
	\textbf{Learning tools:}  &
		\begin{itemize}
			\item Book: Database management systems (3rd edition); authors Ramakrishnan and Gehrke
			\item Book: NO SQL Distilled; authors Sadalage and Fowler
			\item Presentations (in pdf):  on the GitHub repository \url{https://github.com/hogeschool/INFDEV026B}
			\item Assignments, to be done at home (pdf): on the GitHub repository \url{https://github.com/hogeschool/INFDEV026B}
		\end{itemize} \\
	\hline
	\textbf{Connected to \newline competences:} &
		\begin{itemize}
			\item Analysis, design, and realisation of software at level 2
		\end{itemize} \\
	\hline
	\textbf{Learning objectives:} &
		At the end of the course, the student can:
			\begin{itemize}
				\item \textbf{realise} a normalized relational database and implement an application to execute operations on it \texttt{RDBMS,NORM}

				\item \textbf{describe} the differences between relational and non-relational databases \texttt{NONREL}
				
				\item \textbf{use} the map-reduce paradigm to run queries. \texttt{NONREL}

				\item \textbf{describe} models of concurrency and transactions in a modern DBMS \texttt{TRANS}

			\end{itemize} \\
		
	\hline
\end{tabularx}
\newpage

\begin{tabularx}{\textwidth}{|>{\columncolor{lichtGrijs}} p{.26\textwidth}|X|}
	\hline
	\textbf{Content:}&
	\begin{itemize}
		\item relevant concepts in relational databases normalization
		\item fundamental properties of DBMS's: atomicity, consistency, isolation, and durability (ACID)
		\item concurrency and transactions
		\item Map-reduce paradigm.
		\item BASE vs ACID.
		\item Graph databases.
	\end{itemize} \\
	\hline
	\textbf{Course owners:} & \author\\
	\hline
	\textbf{Date:} & \today \\
	\hline
\end{tabularx}
\newpage
