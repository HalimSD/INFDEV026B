\section{Assessment}
	The course is tested with two exams: a practicum check of the practical assignments, and a written exam. The final grade is determined as follows: \\

	0.6 * practicum + 0.4 * written exam

	To pass the exam you must have a positive (i.e. $\geq$ 5.5) grade in both parts.

	\paragraph*{Motivation for grade}
		A professional software developer is required to be able to program code which is, at the very least, \textit{correct}.

		In order to produce correct code, we expect students to show:
		\begin{inparaenum}[\itshape i\upshape)]
			\item a foundation of knowledge about how a programming language actually works in connection with a simplified concrete model of a computer;
			\item fluency when actually writing the code.
		\end{inparaenum}

		The quality of the programmer is ultimately determined by his actual code-writing skills. The quick oral check ensures that each student is able to show that his work is his own and that he has adequate understanding of its mechanisms. The theoretical exam tests that the required foundation of knowledge about databases is also present.


	\subsection{Theoretical examination}
		The general shape of a theoretical exam for the course is made up of a series of highly structured open questions. In each exam the content of the questions will change, but the structure of the questions will remain the same. For the structure of the theoretical exam, see the appendix.


	\subsection{Practical examination}
		The practical assignments are formative and will not be graded. Students are strongly advised to do the assignments to prepare for the pracitcal assessment. The theory part will contain questions about database normalization and transaction concurrency. The practical assessment will contain programming exercises about map-reduce and graph databases. The students must bring a working laptop with them with the necessary tools to run code in C\# and with Neo4j installed.
