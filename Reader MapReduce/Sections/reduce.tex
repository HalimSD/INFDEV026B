Let us consider the query

\begin{lstlisting}[caption = Example query for Reduce, label = code:reduce_query]
SELECT SUM(salary)
FROM employee
WHERE salary > 1500
\end{lstlisting}

\noindent
With what we now have we are able to evaluate the \texttt{WHERE} part and to select some of the attributes but not to compute the sum. We cannot compute the sum with \texttt{Map} because the return type of \texttt{Map} is a collection while the sum returns a single value. We need something able to apply an operation to each element of the collection and accumulate (or aggregate) the result. For this purpose, let us consider first the specific code for two examples, one that computes the concatenation of the string conversion of a sequence of numbers, and the other that sums the salaries of a collection of employees:

\begin{minipage}[t]{0.45\textwidth}
	\begin{lstlisting}[numbers = left, label = code:reduce_comparison_left, caption = Code to concatenate the string representations of numbers]
static string Concat<T>(IEnumerable<T> l)
{
	string c = "";
	for (int i = 0; i < l.Count(); i++)
	{
		c = c + l.ElementAt(i).ToString();
	}
	return c;
}
	\end{lstlisting}
\end{minipage}\hfill
\begin{minipage}[t]{0.45\textwidth}
	\begin{lstlisting}[numbers = left, label = code:reduce_comparison_right, caption = Code to compute the sum of the salaries of the employees]
static double SumSalary(IEnumerable<Employee> employeeTable)
{
	double sum = 0;
	for (int i = 0; i < employeeTable.Count() ; i++)
	{
		sum = sum + employeeTable.ElementAt(i).Salary;
	}
	return sum;
}
	\end{lstlisting}
\end{minipage}

\noindent
Let us try to pinpoint the differences between the two functions and to recycle their pattern (which is the same). Both functions share the following behaviour

\begin{enumerate}[noitemsep]
	\item They initialize a variable containing the result (accumulator).
	\item For all elements of the collection they update the accumulator by applying an operation involving the accumulator and an element of the collection.
	\item They return the final value of the accumulator.
\end{enumerate}

\noindent
First of all, let us look at the types of the method declaration. Listing \ref{code:reduce_comparison_left} returns a \texttt{string} while Listing  \ref{code:reduce_comparison_right} returns \texttt{double}. The type of the argument of Listing \ref{code:reduce_comparison_left} is \texttt{IEnumerable<T>} while the one of Listing \ref{code:reduce_comparison_right} is \texttt{IEnumerable<Employee>}. Thus we can say that our \texttt{Reduce} function returns a generic type \texttt{T2} and takes as input a \texttt{IEnumerable<T1>}. At line 3 both functions initialize an accumulator containing the result with a different value. This value must be passed as input to \texttt{Reduce} because it is specific of the operation we want to execute. This is not enough: at line 6 both functions run a completely different operation that updates the accumulator. Thus, analogously to what we did for \texttt{Map}, we pass as an extra argument a function able to perform the calculation necessary to update the accumulator. So \texttt{Reduce} takes two extra arguments, the initial value of the accumulator, whose type is \texttt{T2}, and the operation to execute. The second argument is a lambda taking as input the accumulator itself and an element of the input collection, and returning the updated accumulator, thus its type is \texttt{Func<T2, T1, T2>}. The definition of \texttt{Reduce} becomes then

\begin{lstlisting}
public static T2 Reduce<T1, T2>(IEnumerable<T1> collection, T2 init, Func<T2, T1, T2> operation)
{
	T2 result = init;
	for (int i = 0; i < collection.Count(); i++)
	{
		result = operation(result, collection.ElementAt(i));
	}
	return result;
}
\end{lstlisting}

\noindent
Let us now implement the two functions above with \texttt{Reduce}. The first one takes as input a collection of numbers and concatenates their string representations. Thus the input will contain a collection of numbers, the accumulator will be set to \texttt{""}, and the lambda will take the accumulator and a number, and add the string representation of the number to the accumulator.

\begin{lstlisting}
string concatenation = MapReduce.Reduce(numbers, "", (accumulator, x) => accumulator + x.ToString());
\end{lstlisting}

In order to compute the sum of the salaries we take as input the collection of employees, we initialize the accumulator to 0.0, and we pass a function that adds to the accumulator each salary.

\begin{lstlisting}
double salarySum = MapReduce.Reduce(employeeTable, 0.0, (accumulator,e) => accumulator + e.Salary);
\end{lstlisting}

\noindent
At this point we are capable of implementing the query in Listing \ref{code:reduce_query}. We first filter the collection with \texttt{Where} and then we compute the sum with \texttt{Reduce}.

\begin{lstlisting}
IEnumerable<Employee> filtered = MapReduce.Where(employeeTable, e => e.Salary > 1500);
double filteredSum = MapReduce.Reduce(filtered, 0.0, (accumulator, e) => accumulator + e.Salary);
\end{lstlisting}

At this point we have \texttt{Map}, \texttt{Reduce}, and \texttt{Where} and we are able to implement a full SQL queries. Moreover we have clearly estabilished that everything works as result of black magic rituals. But at this point a careful reader would have noted that the title of this document is \texttt{Map-Reduce}, and not \texttt{Map-Reduce-Where}. In other words, we have an ``intruder'': the function \texttt{Where}. But would it be possible to implement \texttt{Where} in terms of \texttt{Reduce}? Keep reading and you will find out.

\subsection{Where is Reduce}
In the previous section we showed that \texttt{Map-Reduce-Where} is largely equivalent to SQL. We have also left the open question about whether it is possible to implement \texttt{Where} in terms of \texttt{Reduce}, because the \texttt{Where} method is something extra. We will answer this question right now.

To express \texttt{Where} in terms of \texttt{Reduce} we must define the data structure for the accumulator, its initial value, and the update function for the accumulator. \texttt{Where} returns a new collection containing values filtered from the input according to a condition. Thus the accumulator will be a collection. Its initial value is an empty collection: the filter might remove, as an extreme case, all the elements from the original collection if none satisfies the condition. The lambda takes as input the result collection and each element of the input collection and adds an element if it satisfies the predicate. Thus let us consider again the query in Listing \ref{code:where_query}, which we implement as follows:

\begin{lstlisting}
IEnumerable<Employee> filteredWithReduce =
	MapReduce.Reduce(
		employeeTable,
		new List<Employee>(),
		(queryResult, e) =>
			{
				if (e.Salary > 1500)
					queryResult.Add(e);
				return queryResult;
			});
var data = MapReduce.Map(filteredWithReduce,
	employee =>
		{
			return new
			{
				Name = employee.Name,
				Surname = employee.Surname
			};
		});
\end{lstlisting}

Note that the lambda passed to \texttt{Reduce} checks the condition as well, and adds the result only if the condition is met. This shows that only with \texttt{Map-Reduce} we can build a SQL query. After \texttt{Reduce} is run we use \texttt{Map} to select only the attributes that we want in the result. At this point you would think it is over, but there is more: in the next section we will discover that \texttt{Reduce} is even more powerful.

\subsection{Map is Reduce}
In the previous section we explained how to implement \texttt{Where} using \texttt{Reduce}. But we can do more. We can implement \texttt{Map} with \texttt{Reduce} as well.

The result of \texttt{Map} is a collection, so again the accumulator used in \texttt{Reduce} will be a collection. The initial value of the collection is again the empty collection. The lambda takes each element of the input collection, applies the transformation on each element, and adds it to the accumulator. For instance, the query

\begin{lstlisting}
SELECT name, surname
FROM employee
\end{lstlisting}

\noindent
becomes

\begin{lstlisting}
var dataWithReduce =
	MapReduce.Reduce(employeeTable,
		new List<dynamic>(),
		(queryResult, e) =>
			{
			queryResult.Add(
				new
				{
					Name = e.Name,
					Surname = e.Surname
				});
		return queryResult;
	});
\end{lstlisting}

Note that we have to assign \texttt{dynamic} to the generic argument because we are using an anonymous type for the result, so we cannot know its type at compile time.

The query

\begin{lstlisting}
SELECT name, surname
FROM employee
WHERE salary > 1500
\end{lstlisting}

\noindent
becomes instead

\begin{lstlisting}
var filterAndProjectionWithReduce =
	MapReduce.Reduce(employeeTable,
	new List<dynamic>(),
	(queryResult, e) =>
	{
		if (e.Salary > 1500)
			queryResult.Add(
				new
				{
					Name = e.Name,
					Surname = e.Surname
				});
		return queryResult;
	});
\end{lstlisting}

\noindent
For convenience, we keep using two separate functions, \texttt{Map} and \texttt{Reduce}, but only \texttt{Reduce} will be enough to build a language equivalent to SQL.

\subsection{Join is Reduce}
Up to this point, we are able to use \texttt{Map-Reduce} to implement any SQL query on \underline{one} table. But what if we want to retrieve information frm multiple tables?

Let us consider the example of the same employee table as above, and an additional data structure representing company cars. Each car can be assigned to at most one employee, but a single employee can use more than one company car. The car is represented by a plate number, a model, and a reference to the employee it is assigned to (his ID). Imagine that we want to retrieve the information about the name, surname, and model of car assigned to each employee. The corresponding SQL query would be

\begin{lstlisting}
SELECT e.name, e.surname, c.model
FROM Employee e, Car c
WHERE e.id = c.id
\end{lstlisting}

\noindent
We will now show that only with Map-Reduce it is also possible to implement a join. Recalling the definition of join, this is simply the Cartesian Product of the two tables followed by a filter applied to the result. Remember that the Cartesian Product takes each row from the first table and combines it with all the rows from the second table. Our \texttt{Join} function will thus take two tables and a condition defined as a lambda in the same fashion of what we did for \texttt{Where}. It will then combine each row of the first table with all the rows of the second table and put the combination in the result only if the condition is met.

Let us first give the definition of the class representing the car:

\begin{lstlisting}
class CompanyCar
{
	public string Plate { get; set; }
	public string Model { get; set; }
	public int EmployeeId { get; set; }
	public CompanyCar(string plate, string model, int employee)
	{
		Plate = plate;
		Model = model;
		EmployeeId = employee;
	}
}
\end{lstlisting}

\noindent
Now let us think about the signature of the function for \texttt{Join}: the function takes as input two different tables (or collections) with different element types. Thus the function will have two generic type arguments, \texttt{T1} and \texttt{T2}. The function returns a collection containing the combinations of each row of the first table with all the rows of the second table. Each of these combinations can be represented by a pair containing an element from the first list and an element from the second list. Thus the returned type will be \texttt{IEnumerable<Tuple<T1,T2>>}. \texttt{Join} also takes a condition to filter out some of these combinations. The condition is a lambda that takes a combination and returns a boolean, thus its type is \texttt{Func<Tuple<T1, T2>, bool>}.

The accumulator to use in \texttt{Reduce} will thus be a list containing the combinations created by \texttt{Join} and its initial value is an empty list. In the lambda passed to \texttt{Join} we take each element from the first table and create its combinations with all the rows of the second table.

To do this, let us focus first on how to create the combination of one row of the first table with the rows of the second table. This can be done with \texttt{Reduce} itself by taking as input the second table, initializing the accumulator to en empty list, and adding the pair containing the elements from both the first table and the second if the condition is met. Assuming that \texttt{x} is the variable containing the row from the first table, the following code creates the combination between \texttt{x} and all the rows of the second table, filtered according to the condition:

\begin{lstlisting}
List<Tuple<T1, T2>> combination =
	Reduce(table2, new List<Tuple<T1, T2>>(),
		(c, y) =>
		{
			Tuple<T1, T2> row = new Tuple<T1, T2>(x, y);
			if (condition(row))
				c.Add(row);
			return c;
		});
\end{lstlisting}

\noindent
Now this call can be used in the lambda of the \texttt{Reduce} we use to join the whole table, obtaining:

\begin{lstlisting}[numbers = left, caption = Join with Reduce, label = code:join_with_reduce]
public static IEnumerable<Tuple<T1, T2>> Join<T1, T2>(IEnumerable<T1> table1, IEnumerable<T2> table2, Func<Tuple<T1, T2>, bool> condition)
{
	return
		Reduce(table1, new List<Tuple<T1, T2>>(),
		(queryResult, x) =>
			{
				List<Tuple<T1, T2>> combination =
					Reduce(table2, new List<Tuple<T1, T2>>(),
						(c, y) =>
						{
							Tuple<T1, T2> row = new Tuple<T1, T2>(x, y);
							if (condition(row))
								c.Add(row);
							return c;
						});
				queryResult.AddRange(combination);
				return queryResult;
			});
}
\end{lstlisting}

\noindent
In Listing \ref{code:join_with_reduce} line 4 calls \texttt{Reduce} to perform the whole join, that is the combination of each row from \texttt{table1} with each row of \texttt{table2}. At line 8 we call \texttt{Reduce} to create the combination of a single row from \texttt{table1} with all the rows of \texttt{table2}, filtered according the \texttt{condition} of \texttt{Join}. At line 11 we create the pair containing a row from \texttt{table1} and a row from \texttt{table2}. At line 12 we test the condition and add the combination only if the condition is true. At line 16 we add the combinations created at line 8 to the accumulator.